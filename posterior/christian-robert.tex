\documentclass[a4,12pt]{article}

\usepackage[english]{babel}
\usepackage[utf8]{inputenc}
\usepackage[T1]{fontenc}
\usepackage[proportional]{libertine}
\usepackage[libertine]{newtxmath}

% Import the natbib package and sets a bibliography  and citation styles
\usepackage[numbers]{natbib}
\bibliographystyle{plainnat}
% \setcitestyle{numbers}

\usepackage{amsmath}
\usepackage{graphicx}
\usepackage{hyperref}
\usepackage{paralist}
\usepackage{xcolor}

%%% math
\newcommand{\betap}{\ensuremath{\beta^\prime}}
\newcommand{\data}{\ensuremath{\mathcal{D}}}
\newcommand{\expin}{\ensuremath{\lbrace\mu_i, \sigma_i\rbrace}}
\DeclareMathOperator{\Dirichlet}{Dirichlet}
\newcommand{\given}[2]{\ensuremath{(#1 | #2)}}
\newcommand{\gaussian}{\ensuremath{\mathcal{N}}}
\newcommand{\genbetapr}{\ensuremath{\mathrm{GenBetaPrime}}}
\newcommand{\likelihood}{\ensuremath{\mathcal{L}}}
\newcommand{\lumi}{\ensuremath{\ell_i}}
\newcommand{\Lumi}{\ensuremath{L_i}}
\newcommand{\real}{\ensuremath{{\rm real}}}
\newcommand{\virt}{\ensuremath{{\rm virt}}}
\newcommand{\Nreal}{\ensuremath{N_\real}}
\newcommand{\Nvirt}{\ensuremath{N_\virt}}
\newcommand{\poisson}{\ensuremath{\mathrm{Poisson}}}
\newcommand{\rmdx}[1]{\mbox{d} #1 \,} % differential

\renewcommand{\vec}[1]{\boldsymbol{#1}}
\newcommand{\vecalpha}{\ensuremath{\vec{\alpha}}}
\newcommand{\vecL}{\ensuremath{\vec{L}}}
\newcommand{\vecmu}{\ensuremath{\vec{\mu}}}
\newcommand{\vecnu}{\ensuremath{\vec{\nu}}}
\newcommand{\vecth}{\ensuremath{{\vec{\theta}}}}
\newcommand{\vecu}{\ensuremath{{\vec{u}}}}


\DeclareMathOperator{\erf}{erf}
\newcommand{\mui}{\mu_i}
% \newcommand{\Ni}{N_i}
% \newcommand{\ni}{n_i}
\newcommand{\sigi}{\sigma_i^2}
\DeclareMathOperator{\sgn}{sgn}
\newcommand{\El}{E[l]}
\newcommand{\Vl}{V[l]}

%%% refs
\def \refeq#1{(\ref{eq:#1})}
\def \refsec#1{sec.~\ref{sec:#1}}
\def \refapp#1{app.~\ref{#1}}
\def \reffig#1{fig.~\ref{fig:#1}}

%%% comments
\newcommand{\todo}[1]{{\textsc{\color{red}#1}}}

%%% codes
\newcommand{\tardis}{TARDIS}

\title{Incorporating simulation uncertainty}
\author{Frederik Beaujean}

\begin{document}
\maketitle

\noindent
\emph{I here try to formulate the problem in a way that abstracts
  nearly all the physics. Discussing with you helped a lot and I think
  I straightened it out now. I work with the supernova experts at the
  ESO and I am responsible for the statistics part.}

\section*{Overview}
My goal is to do a fit comparing a luminosity spectrum obtained from a
telescope to the model prediction. The spectrum describes the
intensity of light of various wavelenghts emitted by a distant
supernova. Our parameters of interest $\vecth$ include the abundance
of various elements produced during the supernova explosion. These
elements produce dips and other absorption features in the
spectrum.

The main difficulty is that I cannot compute the luminosity
$L(\vecth)$ directly. I have to start an expensive simulation to
compute the spectrum for the $\sim 2000$ bins. The simulation is
multithreaded and takes about 100s for a single evaluation on the
16-core machines at our C2PAP compute cluster. I was told that some
parts are not trivially parallel so it doesn't make much sense to
scale the simulation to multiple machines. Our simulation is a much
simplified version of a full-blown code that takes weeks on hundreds
of cores. Ultimately our simulation propagates $N$ photons through the
supernova. The simulation uncertainty on $L$ would tend to zero for $N
\to \infty$. But we can only afford $N \approx 40000$.

It seems to me this is a general problem a lot of people have so I
searched for a Bayesian solution but found only some frequentist
recipes that require a lot of simulated photons in each wavelenght
bin. But I want to be able to treat the case with low counts as well.

\section*{Statistical model}

Bayes' theorem

\begin{align}
  P\given{\vecth}{\data} \propto P\given{\data}{\vecth} P(\vecth)
\end{align}
For most of $\vecth$ we have relevant prior information but I won't
discuss $P(\vecth)$ further. Regarding $P\given{\data}{\vecth}$, I
have to state that we don't have the actual data $\data$, that is the
counts etc. We only get the likelihood as a function of $L$,
$P\given{\data}{L}$.

In our model, the true luminosity is a given function $f(\vecth)$ that
we cannot evaluate. But for a given random-number seed $\xi$, our
simulation represented by $g(\vecth, \xi)$ yields an estimator of
$L$. I want to marginalize over $\xi$ to take into account the
uncertainty from the simulation.
\begin{align}
  \label{eq:xi-int}
  P\given{\data}{\vecth} = \int \rmdx{\xi}  P(\xi) P\given{\data}{\vecth, \xi}
\end{align}
Formally I change variables from $\xi$ to $L=g(\vecth, \xi)$ for fixed $\vecth$
\begin{align}
  P\given{\data}{\vecth} = \int \rmdx{L} \left(\frac{\rmdx{g(\vecth, \xi)}}{\rmdx{\xi}} \right)^{-1} P(\xi(L)) P\given{\data}{\vecth, L}
\end{align}
Now I combine the first terms as a distribution over $L$ for fixed
$\vecth$ and I use the fact that $P\given{\data}{\vecth, L}$ doesn't
explicitly depend on $\vecth$ to obtain the final expression
\begin{align}
  \label{eq:L-int}
  P\given{\data}{\vecth} = \int \rmdx{L} P(L| \vecth) P\given{\data}{L}
\end{align}
I can incorporate the distribution from the telescope
$P\given{\data}{L}$ but now I need to account for the uncertainty from
the simulation through $P(L|\vecth)$. For $N \to \infty$, $P(L| \vecth) \to \delta(L -
L(\vecth))$.

What is won?  but A naive approach would be to approximate
\refeq{xi-int} by repeating the simulation for many $\xi$ but that's
expensive and we want to avoid it. Instead, I want to put in my
knowledge about the simulation to get an approximation for $P(L|
\vecth)$ to either solve \refeq{L-int} analytically or by 1D
quadrature which is much faster because both distributions under the
integral sign are assumed much faster to evaluate than $g(\vecth,
\xi)$.

\end{document}
