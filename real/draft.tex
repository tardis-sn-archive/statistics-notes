%%%%%%%%%%%%%%%%%%%%%%%%%%%%%%%%%%%%%%%%%%%%%%%%%%%%%%%%%%%%%%%%%%%%%%%%%%%%%
%
% Bayesian prediction of luminosity distribution based on one simulation run:
% An example from TARDIS
%
%%%%%%%%%%%%%%%%%%%%%%%%%%%%%%%%%%%%%%%%%%%%%%%%%%%%%%%%%%%%%%%%%%%%%%%%%%%%%
\documentclass[11pt]{article}
\usepackage[utf8]{inputenc}

\usepackage[a4paper, left=20mm, right=20mm, top=25mm, bottom=25mm]{geometry}
\usepackage[numbers]{natbib}
\usepackage{hyperref}
\synctex=1

\usepackage[colorinlistoftodos]{todonotes} % Use 'disable' to remove todos in final version
\newcommand{\fred}[1]{\todo[color=orange!40,inline]{#1}} %
\newcommand{\checked}{\todo[color=green,noline]{\checkmark}} %
\newcommand{\checkedl}{\todo[color=brown]{\checkmark}} %
\newcommand{\hans}[1]{\todo[color=yellow!30,inline]{#1}} %
\newcommand{\hanslong}[2][]{\todo[%bordercolor=red,
  color=white,inline,caption={2do}, #1]{
    \begin{minipage}{\textwidth} #2\end{minipage}}} %

%% CHANGE THIS DATE AS YOU MODIFY THE FILE
\newcommand{\vdate}{160510}
\pagestyle{myheadings}
\markboth{Version \vdate} {Version \vdate}
\parindent=0mm
\def\pp{\hfill\par \vspace{\baselineskip}}
%% \def\pp{}
%%%%%%%%%%%%%%%%%%%%%%%%%%%%%%%%%%%%%%%%%%%%%%%%%%%%%%%%%%%%%%%%%%%%%%%%%%%%%
\usepackage{bm,amssymb,amsfonts,amsmath}  % bold math symbols, AMStex
\usepackage{graphicx}    % standard graphics
%%%%%%%%%%%%%%%%%%%%%%%%%%%%%%%%%%%%%%%%%%%%%%%%%%%%%%%%%%%%%%%%%%%%%%%%%%%%%
\newcommand{\lleq}[1]{\label{#1} }
% TO REMOVE THE LABEL LETTERS FROM THE EQUATION DISPLAY, COMMENT OUT THIS LINE:
\renewcommand{\lleq}[1]{\label{#1} {\scriptstyle {\rm (#1)}} \hspace*{2ex} }
% \renewcommand{\labelenumi}{\arabic{section}.\arabic{subsection}.\arabic{enumi}}
% \renewcommand{\thesubsubsection}{\arabic{subsubsection}}

\newcounter{mnumi}
\newenvironment{mnumerate}{
  \begin{list}{\arabic{mnumi}.}
    {\usecounter{mnumi} \setlength{\itemsep}{0pt}
      \setlength{\leftmargin}{3ex}
    }
  }
  {\end{list}}

\newenvironment{mtemize}{
  \begin{list}{$\bullet$}
    {\setlength{\itemsep}{0pt}
     \setlength{\leftmargin}{3ex}
    }
  }
  {\end{list}}

\newcommand{\hmod}  {{\mathcal{H}}}  % Model
\newcommand{\ldef}{\;{:}{=}\;}
\newcommand{\rdef}{\;{=}{:}\;}
\newcommand{\realnumbers}{\mathbb{R}}
\newcommand{\integers}{\mathbb{Z}}
\newcommand{\cond}{\,|\,}
\newcommand{\NBD}{\mathrm{NB}}
\newcommand{\var}{\mathrm{var}}

\newcommand{\smA}{{\scriptscriptstyle A}}
\newcommand{\smB}{{\scriptscriptstyle B}}
\newcommand{\smH}{{\scriptscriptstyle H}}
\newcommand{\smK}{{\scriptscriptstyle K}}
\newcommand{\smL}{{\scriptscriptstyle L}}
\newcommand{\smM}{{\scriptscriptstyle M}}
\newcommand{\smN}{{\scriptscriptstyle N}}
\newcommand{\smR}{{\scriptscriptstyle R}}

\newcommand{\smo}{{\scriptscriptstyle 1}}
\newcommand{\smt}{{\scriptscriptstyle 2}}
\newcommand{\smr}{{\scriptscriptstyle 3}}
\newcommand{\smf}{{\scriptscriptstyle 4}}
\newcommand{\smv}{{\scriptscriptstyle 5}}
\newcommand{\smx}{{\scriptscriptstyle 6}}
\newcommand{\sms}{{\scriptscriptstyle 7}}
\newcommand{\sme}{{\scriptscriptstyle 8}}
\newcommand{\smn}{{\scriptscriptstyle 9}}

\newcommand{\bma}{{\bm{a}}}
\newcommand{\bmb}{{\bm{b}}}
\newcommand{\bmc}{{\bm{c}}}
\newcommand{\bmd}{{\bm{d}}}
\newcommand{\bme}{{\bm{e}}}
\newcommand{\bmf}{{\bm{f}}}
\newcommand{\bmg}{{\bm{g}}}
\newcommand{\bmh}{{\bm{h}}}
\newcommand{\bmi}{{\bm{i}}}
\newcommand{\bmj}{{\bm{j}}}
\newcommand{\bmk}{{\bm{k}}}
\newcommand{\bml}{{\bm{\ell}}}
\newcommand{\bmm}{{\bm{m}}}
\newcommand{\bmn}{{\bm{n}}}
\newcommand{\bmo}{{\bm{o}}}
\newcommand{\bmp}{{\bm{p}}}
\newcommand{\bmq}{{\bm{q}}}
\newcommand{\bmr}{{\bm{r}}}
\newcommand{\bms}{{\bm{s}}}
\newcommand{\bmt}{{\bm{t}}}
\newcommand{\bmu}{{\bm{u}}}
\newcommand{\bmv}{{\bm{v}}}
\newcommand{\bmw}{{\bm{w}}}
\newcommand{\bmx}{{{\bm{x}}}}
\newcommand{\bmy}{{\bm{y}}}
\newcommand{\bmz}{{\bm{z}}}
\newcommand{\bmA}{{\bm{A}}}
\newcommand{\bmB}{{\bm{B}}}
\newcommand{\bmC}{{\bm{C}}}
\newcommand{\bmD}{{\bm{D}}}
\newcommand{\bmE}{{\bm{E}}}
\newcommand{\bmF}{{\bm{F}}}
\newcommand{\bmG}{{\bm{G}}}
\newcommand{\bmH}{{\bm{H}}}
\newcommand{\bmI}{{\bm{I}}}
\newcommand{\bmJ}{{\bm{J}}}
\newcommand{\bmK}{{\bm{K}}}
\newcommand{\bmL}{{\bm{L}}}
\newcommand{\bmM}{{\bm{M}}}
\newcommand{\bmN}{{\bm{N}}}
\newcommand{\bmO}{{\bm{O}}}
\newcommand{\bmP}{{\bm{P}}}
\newcommand{\bmQ}{{\bm{Q}}}
\newcommand{\bmR}{{\bm{R}}}
\newcommand{\bmS}{{\bm{S}}}
\newcommand{\bmT}{{\bm{T}}}
\newcommand{\bmU}{{\bm{U}}}
\newcommand{\bmV}{{\bm{V}}}
\newcommand{\bmW}{{\bm{W}}}
\newcommand{\bmX}{{\bm{X}}}
\newcommand{\bmY}{{\bm{Y}}}
\newcommand{\bmZ}{{\bm{Z}}}

\newcommand{\bmalpha}{{\bm{\alpha}}}
\newcommand{\bmbeta}{{\bm{\beta}}}
\newcommand{\bmchi}{{\bm{\chi}}}
\newcommand{\bmdelta}{{\bm{\delta}}}
\newcommand{\bmepsilon}{{\bm{\varepsilon}}}
\newcommand{\bmphi}{{\bm{\phi}}}
\newcommand{\bmgamma}{{\bm{\gamma}}}
\newcommand{\bmeta}{{\bm{\eta}}}
\newcommand{\bmiota}{{\bm{\iota}}}
\newcommand{\bmkappa}{{\bm{\kappa}}}
\newcommand{\bmlambda}{{\bm{\lambda}}}
\newcommand{\bmmu}{{\bm{\mu}}}
\newcommand{\bmnu}{{\bm{\nu}}}
\newcommand{\bmomega}{{\bm{\omega}}}
\newcommand{\bmpi}{{\bm{\pi}}}
\newcommand{\bmpsi}{{\bm{\psi}}}
\newcommand{\bmrho}{{\bm{\rho}}}
\newcommand{\bmsigma}{{\bm{\sigma}}}
\newcommand{\bmtau}{{\bm{\tau}}}
\newcommand{\bmtheta}{{\bm{\theta}}}
\newcommand{\bmupsilon}{{\bm{\upsilon}}}
\newcommand{\bmxi}{{\bm{\xi}}}
\newcommand{\bmzeta}{{\bm{\zeta}}}

\newcommand{\refeq}[1]{Eq.~(\ref{#1})}
\newcommand{\reffig}[1]{Fig.~\ref{fig:#1}}

\DeclareMathOperator{\GammaDist}{Gamma}
\newcommand{\Kalpha}{{K_\alpha}}
\newcommand{\Kbeta}{{K_\beta}}
\newcommand{\npack}{n_p}
\newcommand{\lmax}{\ell_{\rm max}}
\newcommand{\Lmax}{L_{\rm max}}
\newcommand{\rmdx}[1]{\mbox{d} #1 \,} % differential
\newcommand{\firstDeriv}[1]{\frac{\partial}{\partial #1}}
\newcommand{\secDeriv}[1]{\frac{\partial^2}{\partial #1^2}} % second partial derivative
\newcommand{\secPartial}[2]{\frac{\partial^2}{\partial #1 \, \partial #2}} % second partial derivative
\newcommand{\tardis}{TARDIS}

%%%%%%%%%%%%%%%%%%%%%%%%%%%%%%%%%%%%%%%%%%%%%%%%%%%%%%%%%%%%%%%%%%%%%%%%%%%%%
\begin{document}

\begin{center}
  \textbf{\Large Bayesian prediction of supernova luminosity  distributions}\\[8pt]
  \textbf{\Large based on simulation output: An example from \tardis}\\[12pt]
\end{center}

\textbf{Abstract} TODO\\

\section{Introduction} \label{sec:intro}

\section{Notation} \label{sec:notation}

We consider a computer simulation that, given input parameters
$\bmtheta$ and a random-number seed, produces a set of simulated data
$ S = \{ (\nu_j, \ell_j): j=1\dots\npack \}$ consisting of $\npack$
packets, each with a frequency and a luminosity. We distinguish
between a random variable $L$ and its concrete realization $\ell$.  To
learn about $\bmtheta$, we need to compare to data from a telescope
available as luminosities in frequency bins, $D = \{ ( \nu_b, \ell_b):
b=1 \dots B\}$. The natural estimate from a single simulation is to
add up the luminosities of all packets in a bin.  With a different
seed but identical $\bmtheta$, both the number of packets in the bin
and their luminosities could be different. Our goal is to estimate the
distribution based on a \emph{single} simulation without recourse to
expensive repetitions or asymptotic approximations; i.e., we seek
$P(L_b \cond \nu_b, \npack, \bmnu, \bml )$. In an actual physics
analysis, one would proceed to plug in the observed value $\ell_b$ and
perform an analysis to learn about $\bmtheta$. But here we focus on
the specific problem of $P(L_b \cond \nu_b, \npack, \bmnu, \bml )$ and
hence omit $\bmtheta$ throughout.

Various features, for example absorption lines, usually make it
difficult to explicitly model the spectrum---the luminosity as a
function of frequency. In a binned analysis, we can avoid these
difficulties associated to modeling $n_b$. Our approach, however,
requires an explicit expression of how the distribution of the
luminosity of a single packet in our simulation changes with $\nu$. It
is this crucial piece of information that allows us to extend previous
methods \fred{Refs: Zech, maybe others, too} in dealing with the
extreme cases when the number of packets in a bin, $n_b$, is one or
even zero.

\section{Model} \label{sec:model}

We consider the total number of packets $\npack$ fixed, for example
chosen by a user. In contrast, the number of packets in a bin, $n_b$
is a realization of a random variable $N_b$ which we assume to follow
a Poisson distribution. With this assumption, we neglect bin-by-bin
correlations and treat each bin independently which allows us to focus
on a single bin and to drop the subscript $b$ in the remainder. Hence
\begin{align}
  \lleq{caa}
  p(N\cond\lambda) &= \frac{e^{-\lambda}\lambda^N}{N!}
  \qquad N = 0,1,\ldots,\infty ,
\end{align}
where the dependency of the Poisson mean $\lambda>0$ on $\nu$ is of no
concern to us. Let $\bmL$ denote the collection of luminosities of
packets in a certain bin. The total luminosity in that bin is then
given by
\begin{align}
  \lleq{cab}
  L = \sum_{j=1}^{N} L_j\,.
\end{align}
We assume that each $L_j$ has a distribution depending on $\nu$ and
other parameters $\bmphi$ and that each packet is statistically
independent. Since $N$ is random, $L$ is said to follow a compound
Poisson process. The joint distribution then is
\begin{align*}
  \label{cac}
  (N,\bmL) &\ \sim\ p(N,\bmL\cond \bmphi,\lambda, \nu)
  = p(\bmL\cond N,\bmphi, \nu)\; P(N\cond \lambda) = \prod_{j=1}^N p(L_j\cond \bmphi, \nu) \; P(N\cond \lambda),
\end{align*}
where we used that the Poisson term does not depend on $\nu$
explicitly. In general, the distribution of $L_j$ could be a of a
different functional form in each bin. For simplicity, we assume it is
identical but may depend on parameters that differ from bin to bin. In
practice, the distribution of packets does not change abruptly with
frequency. We therefore assume that any parameter needed to fully
specify $p(L_j)$ is given explicitly as a continuous function of $\nu$
and the unknown parameters $\bmphi$. This allows us to use a small
number of parameters to describe $P(L)$ across the entire frequency
range and in turn lets us determine $\bmphi$ from fitting all
simulated packets simultaneously whereas $\lambda$ is given only by
the packets in a bin.

Our goal of predicting $L$ in bin $b$ given the simulation output
$(\npack,\bml, \bmnu)$ is captured in the probability $p(L\cond \nu,
\npack,\bml, \bmnu)$ with all relevant parameters integrated out using
the law of total probability.  To simplify the treatment, we assume
that the frequency bins are narrow or equivalently that $p(L \cond
\nu, \npack, \bml, \bmnu)$ does not vary within a bin. This implies
that all every sample $L_j$ in the bin follows the same distribution.
As the reference frequency, we choose the bin center.

We first note from \refeq{caa} that $N$ does not depend on $\bml$ at
all and only indirectly depends on $\bmnu$ through $n$. Since $L$ is
fully determined by $(N,\bmL)$, it does not depend on
$(\npack,\bml,\bmnu)$, i.e.
\begin{align}
  \lleq{pmi}
  p(L\cond N,\bmL,\nu,\npack,\bml, \bmnu)
  = p(L\cond N, \bmL)
  &= \delta(L - \textstyle\sum_{j=1}^N L_j),
\end{align}
\fred{Uniform order: $\nu$ first, then $\ell$, compare to $S$}
where the $L_j$ are assumed to occur at $\nu$.  Using the law of total
probability repeatedly, we can write the desired distribution
\fred{Discuss $N=0$ case somewhere}
\begin{align}
  \lleq{pmj}
  p(L\cond \nu, \npack,\bml, \bmnu)
  &= \sum_N p(N\cond \nu, \npack,\bml, \bmnu) \, p(L\cond N,\nu, \npack,\bml, \bmnu) \\
  \lleq{pmjb}
  &= \sum_N p(N\cond n)\,\int \rmdx{\bmL} p(L\cond N,\bmL,\nu, \npack,\bml, \bmnu)\, p(\bmL\cond N,\nu, \npack,\bml, \bmnu)
  \\
  \lleq{pmjc}
  &= \sum_N p(N\cond n)\,\int \rmdx{\bmL} \delta(L - \textstyle\sum_{j=1}^N L_j)
  \, p(\bmL\cond N,\nu, \npack,\bml, \bmnu)
\end{align}
The dependence on the simulation output $(\npack,\bml, \bmnu)$ enters
through $p(\bmL\cond N,\npack,\bml, \bmnu)$ in terms of the parameters
$\bmphi$ of $p(\bmL\cond N,\bmphi)$,
\begin{align}
  p(\bmL\cond N,\nu, \npack,\bml, \bmnu)
  &= \int \rmdx{\bmphi} p(\bmL\cond \bmphi,N, \nu, \npack,\bml, \bmnu)\,p(\bmphi\cond N,\nu, \npack,\bml, \bmnu) \\
  \lleq{pmk}
  &= \int \rmdx{\bmphi} p(\bmL\cond \bmphi, N, \nu)\,p(\bmphi\cond N,\nu, \npack,\bml, \bmnu).
\end{align}
because $\bmL$ is fully determined by $N,\bmphi, \nu$ and hence
independent of $\npack,\bml, \bmnu$.  With Bayes' theorem we learn about
$\bmphi$ from the simulation output
\begin{align}
  \lleq{pmkb}
  p(\bmphi\cond N,\nu, \npack,\bml, \bmnu) &=
  \frac{p(\bml\cond \bmphi, N,\nu, \npack,\bmnu)\,p(\bmphi\cond N,\nu, \npack,\bmnu)} %
  {\int \rmdx{\bmphi} p(\bml\cond \bmphi,N,\nu, \npack,\bmnu)\,p(\bmphi\cond N,\nu, \npack,\bmnu)}.
\end{align}
By using a multiplicity- and frequency-independent prior for $\bmphi$,
\begin{align}
  \lleq{pml}
  p(\bmphi\cond N,\nu, \npack,\bmnu) &= p(\bmphi),
\end{align}
and by recognizing that $\bml$ is independent of $N$ and $\nu$
\begin{align}
  \lleq{pmla}
  p(\bml\cond \bmphi,N,\nu, \npack,\bmnu) = p(\bml\cond \bmphi,\npack,\bmnu)
\end{align}
we can write \refeq{pmk} as
\begin{align}
  \lleq{pmn}
  p(\bmL\cond N,\nu, \npack,\bml, \bmnu)
  &= \int \rmdx{\bmphi} p(\bmL\cond N,\bmphi)\, %
  \frac{p(\bml\cond \bmphi, \npack, \bmnu)\,p(\bmphi)} %
  {\int \rmdx{\bmphi} p(\bml\cond \bmphi,\npack,\bmnu)\,p(\bmphi)} \\
  \lleq{pmo}
  &= \frac{1}  {p(\bml\cond \npack, \bmnu)} %
  \int \rmdx{\bmphi} p(\bmL\cond \bmphi, N)\, %
  p(\bml\cond \bmphi, \npack,\bmnu)\,p(\bmphi)
\end{align}
with $p(\bml\cond \npack, \bmnu) = \int \rmdx{\bmphi} p(\bml\cond
\bmphi, \npack,\bmnu)\,p(\bmphi)$ is the evidence in the denominator.
Inserting this into (\ref{pmjc}), we obtain the desired master equation:
\begin{equation}
  \lleq{pmp}
  \boxed{
  p(L\cond \nu, \npack,\bml, \bmnu)
  = \frac{1}{p(\bml\cond \npack, \bmnu)}
  \sum_N p(N\cond n)\,\int \rmdx{\bmL} \delta(L - \textstyle\sum_j L_j)
  \displaystyle\int \rmdx{\bmphi} p(\bmL\cond \bmphi, N, \nu) \, p(\bml\cond \bmphi,\npack,\bmnu)
  \, p(\bmphi)
  }
\end{equation}
with
\begin{align}
  \lleq{pmq}
  p(\bmL\cond \bmphi, N, \nu) &= \prod_{j=1}^N p(L_j\cond \bmphi, \nu), \\
  \lleq{pmr}
  p(\bml\cond \bmphi,\npack,\bmnu) &= \prod_{j=1}^{\npack} p(\ell_j\cond \bmphi, \nu_j)
\end{align}
and $p(\cdot \cond \bmphi, \cdot)$ is the same function in both
equations.  In the next section, we compute the general result for
$P(N \cond n)$. Equation \refeq{pmp} implies that once we specify the
prior $p(\bmphi)$ and the likelihood $p(L_j\cond \bmphi)$, we have a
prediction for $L$ which takes into account all possible $N$,
$\lambda$ and $\bmphi$. Both choices depend on the simulation at hand
but a general consideration to render our approach tractable is that
the distribution of a single packet luminosity $P(L_j \cond \bmphi,
\nu)$ should be such that the integral over $\bmL$ has a closed-form
solution. In other words, the distribution of $L = \sum_N L_j$ should
be known.
%  An important special case is the class of \emph{stable}
% distributions.

\subsection{Predicting the number of packets in a bin}
\fred{Show reference Poisson analysis here of before?}

We use Bayes' theorem to find the posterior distribution for $\lambda$
given data $n$ and the Poisson hypothesis
\begin{align}
  \lleq{pmd}
  p(\lambda\cond n) %
  &= \frac{p(n\cond\lambda)\,p(\lambda)} {p(n)}
  \ =\ \frac{p(n\cond\lambda)\,p(\lambda)} %
  {\int \rmdx{\lambda} p(n\cond\lambda)\,p(\lambda)}
\end{align}
to find the posterior predictive of $N$ given $n$ taking into account
all possible values of $\lambda$,
\begin{align}
  \lleq{pme}
  p(N\cond n)
  &= \int \rmdx{\lambda}p(N,\lambda\cond n)
  \ =\ \int \rmdx{\lambda} p(N\cond\lambda, n)\,p(\lambda\cond n)
  \ =\ \frac{\int \rmdx{\lambda} p(N\cond\lambda)\,p(n\cond\lambda)\,p(\lambda)}
  {\int \rmdx{\lambda} p(n\cond\lambda)\,p(\lambda)}
\end{align}
where $p(N\cond\lambda,n) = p(N\cond\lambda)$ is independent of $n$.
A common choice is the prior
\begin{align}
  \lleq{pmf}
  p(\lambda) = c\lambda^{-a},
\end{align}
where $a=0$ gives the uniform prior\footnote{For a uniform prior, the
  case $a{=}0$, we require of course a maximum $\lambda_{\rm max}$. As
  long as $\lambda_{\rm max}$ is large enough, the integral can
  effectively be taken to infinity.} and $a=1/2$ gives the Jeffreys
and reference prior. Both choices lead to an improper prior but the
posterior predictive is proper provided that $n>0$ or $N>0$.  The
standard results for the evidence and the posterior predictive are
\begin{align}
  \lleq{pmg}
  p(n) &= %
  \int_0^\infty d\lambda\, p(n\cond\lambda)\,p(\lambda)
  \ =\
  c \int_0^\infty d\lambda\, \frac{e^{-\lambda}\,\lambda^{n-a}}{n!}
  \ =\ c\frac{(n-a)!}{n!}
  \\
  \lleq{pmh}
  p(N\cond n) %
  &= \frac{n!}{(n-a)!} \int_0^\infty d\lambda\,
  \frac{e^{-2\lambda} \lambda^{N+n-a}}{N!\, n!}
  \ =\ \frac{(N+n-a)!}{N!\,(n-a)!\,2^{N+n+1-a}}
\end{align}
which we recognise as a Negative Binomial distribution describing the
probability to observe $N$ successes given $k=n+1-a$ failures in a
sequence of Bernoulli trials with success probability $\rho=1/2$
\begin{align}
  \lleq{pmhb}
  p(N\cond n) &= \NBD(N\cond k{=}n{+}1{-}a, \rho{=}\tfrac{1}{2})
  \ =\ \binom{N{+}n{-}a}{N}
  \left(\frac{1}{2}\right)^N \left(\frac{1}{2}\right)^{n+1-a}
\end{align}
For non-integer $a$, the Negative Binomial generalizes immediately to
the Pólya distribution by replacing $x! \to \Gamma(x+1)$. Note that
the distribution of $N$ given $n$ is independent of $\bmL$ and $\bml$
by construction but depends on $\bmnu$ implicitly through $n$.

\section{Example application: \tardis} \label{sec:example}

The \tardis{} packages simulates supernovae with \fred{Wolfgang's
  favorite description of his package}. Here we focus on the real
packets as opposed to the virtual packets output by tardis.

Based on staring at the distribution of luminosities,

We use the shorthand $\alpha_n \equiv \alpha(\nu_n \cond
\bmphi), \beta_n \equiv \beta(\nu_n \cond \bmphi), \alpha \equiv
\alpha(\nu \cond \bmphi), \beta \equiv \beta(\nu \cond \bmphi)$.

\subsection{Gradients} \label{sec:gradients}

\begin{align}
  \label{eq:grad-posterior}
  \left( \firstDeriv{\alpha_i}, \firstDeriv{\beta_i} \right) \log P(\bmx \cond \alpha, \beta) &=
  \left(
    \sum_n \left[ \log \beta_n - \Psi(\alpha_n) + \log x_n \right] \nu_n^i \;,
    \sum_n \left[ \frac{\alpha_n}{\beta_n} - x_n \right]\nu_n^i
  \right)
\end{align}

\begin{align}
  \label{eq:grad-prediction}
  \left( \firstDeriv{\alpha_i}, \firstDeriv{\beta_i}, \firstDeriv{N} \right) \log P(X \cond N \alpha, \beta)  &=
  \left(
    \left[ \log \beta - \Psi(N \alpha) + \log X \right] N \nu^i,
    \left[ \frac{N\alpha}{\beta} - X \right]\nu^i,
    \left[ \log \beta -\Psi(N \alpha) + \log X \right] \alpha
  \right)
\end{align}

\begin{align}
  \label{eq:grad-nb}
  \firstDeriv{N} \log P(N \cond n) = \Psi(N+n-a+1) - \Psi(N+1) - \log 2
\end{align}

\subsection{Hessians} \label{sec:hessians}

\begin{align}
  \label{eq:hess-posterior}
    - \left(
    \begin{array}{ccc}
      \secPartial{\alpha_i}{\alpha_j} & \secPartial{\alpha_i}{\beta_j}\\
      \dots & \secPartial{\beta_i}{\beta_j}
    \end{array}
  \right) \log P(\bmx \cond \alpha, \beta)
    &= \left(
    \begin{array}{cc}
      \sum_n \Psi'(\alpha_n) \nu_n^{i+j} & -\sum_n \frac{\nu_n^{i+j}}{\beta_n}\\
      \dots & \sum_n \frac{\alpha_n}{\beta_n^2} \nu_n^{i+j}
    \end{array}
  \right)
\end{align}
with $\alpha_n \equiv \alpha(\nu_n \cond \bmphi), \beta_n \equiv \beta(\nu_n \cond \bmphi)$.

 \begin{align}
  \label{eq:hess-prediction}
  & - \left(
    \begin{array}{ccc}
      \secPartial{\alpha_i}{\alpha_j} & \secPartial{\alpha_i}{\beta_j} & \secPartial{\alpha_i}{N}\\
      \dots & \secPartial{\beta_i}{\beta_j} & \secPartial{\beta_i}{N}\\
      \dots & \dots & \secDeriv{N}
    \end{array}
  \right) \log P(X \cond N \alpha, \beta)
  \\
  &=
  \left(
    \begin{array}{ccc}
      N^2 \Psi'(N \alpha) \nu^{i+j} & -\frac{N \nu^{i+j}}{\beta} & \left[ -\log \beta +\Psi(N \alpha) + N \alpha \Psi'(N \alpha) - \log X \right] \nu^{i}\\
      \dots &  \frac{N\alpha}{\beta^2} \nu^{i+j} & -\frac{\alpha}{\beta} \nu^{i} \\
      \dots & \dots & \alpha^2 \Psi'(N \alpha)
    \end{array}
  \right)
\end{align}
with $\alpha \equiv \alpha(\nu \cond \bmphi), \beta \equiv \beta(\nu \cond \bmphi)$.
\begin{align}
  \label{eq:hess-nb}
   - \secDeriv{N} \log P(N \cond n) = \Psi'(N+1) - \Psi'(N+n-a+1)
\end{align}

\section{Conclusion} \label{sec:conclusion}

\end{document}

% Local Variables:
% compile-command:"rubber --pdf -W refs -S draft"
% End:
